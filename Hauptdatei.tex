\documentclass[a4paper,12pt,twoside]{article}

\usepackage[ngerman]{babel} % Deutsche Sonderzeichen und Silbentrennung nutzen
\usepackage[german]{nomencl} % Erstellen von Listen (Nomenklaturen)
\usepackage{fontspec} % Zeichenencoding
\usepackage{graphicx} % Grafiken aus PNG Dateien einbinden
\usepackage[right]{eurosym} % Eurozeichen einbinden
\usepackage{longtable} % mehrseitige Tabellen ermöglichen
\usepackage{color} % Paket für Textfarben
\usepackage{amssymb} % Mathematische Symbole importieren
\usepackage{array} % für Tabellen
\usepackage{blindtext} % Um Blindtext als Beispiel zu erstellen
\usepackage[dvipsnames,table,xcdraw]{xcolor}
\usepackage{amsmath} % Für Formeln und Formelverzeichnis
\usepackage[backend=biber, sorting=none, citestyle=ieee]{biblatex} % Für Zitate
\usepackage{csquotes} % Für babel
\usepackage{url} % bricht lange URLs "schön" um
\usepackage[hidelinks]{hyperref} % Für Links intern und extern
\usepackage{multirow} % Tabelle mit zusammengefügten Zeilen
\usepackage{wrapfig} % Textumflossene Grafik
\usepackage{caption} % Passe deine Bild- Tabellen-, etc. -überschriften an. 
\usepackage{subcaption} % Zum Einbinden von Subfigs und Subcaptions

% Ohne Einrückung bei neuem Absatz
\setlength{\parindent}{0pt}

% Festlegung Art der Zitierung
\addbibresource{Quellen.bib}

% Setup Captions
\captionsetup[wrapfigure]{format=plain}

% Begin Dokument =============================================================
\begin{document}

\pagenumbering{Roman}
\setcounter{page}{1}

% Inhaltsverzeichnis
\setcounter{tocdepth}{2} % Nur Section und Subsection im Inhaltsverzeichnis anzeigen
\tableofcontents
\clearpage

% Abbildungsverzeichnis
\addcontentsline{toc}{section}{Abbildungsverzeichnis} % Abbildungsverzeichnis im Inhaltsverzeichnis anzeigen
\listoffigures
\clearpage

% Tabellenverzeichnis
\addcontentsline{toc}{section}{Tabellenverzeichnis} % Tabellenverzeichnis im Inhaltsverzeichnis anzeigen
\listoftables
\clearpage

% Begin Des Inhalts ==========================================================
\pagenumbering{arabic}

\section{Erste Ebene}\label{sec:ebene}
\blindtext
\subsection{Zweite Ebene}\label{sec:subebene}
\blindtext[1]
\subsubsection{Dritte Ebene}\label{sec:subsubebene}
Ich referenziere hier auf Kapitel \ref{sec:ebene}.
\blindtext[2]
\clearpage

\section{Umgebungen}\label{sec:umgebungen}
Hier sind einige Beispiele für Umgebungen aufgelistet.

\subsection{Umgebung für Listen}
\begin{itemize}
    \item Eintrag A
    \item Eintrag B
    \item \dots
\end{itemize}

\subsection{Umgebung für Tabellen}
Bei der Erstellung der Tabellen gibt es viele hilfreiche Tools im Internet
(\href{https://www.tablesgenerator.com/}{Table Generator}).

Dies Tabelle \ref{table:simple} wurde mit dem Table Generator erstellt.
\begin{table}[h!]
    \centering
    \caption{So muss es in der Studienarbeit aussehen}
    \label{table:simple}
    \begin{tabular}[t]{|l|l|l|}
        \rowcolor[HTML]{CBCEFB}
        \hline 
        A & B & C \\ \hline
        1 & 2 & 3 \\
        1 & 2 & 3 \\ \hline
    \end{tabular}
\end{table}

Die Tabelle \ref{table:multirow} besitzt sogar verbundene Zellen.
\begin{table}[h!]
    \centering
    \caption{Mit verbundenen Zellen}
    \label{table:multirow}
    \begin{tabular}[t]{|l|l|l|}
    \hline
                             & \multicolumn{2}{l|}{Eigenschaften} \\ \hline
    \multirow{3}{*}{Objekte} & Tag              & Name            \\ \cline{2-3} 
                             & Frage            & Datum           \\ \cline{2-3} 
                             & Author           & Name            \\ \hline
    \end{tabular}
\end{table}

\clearpage

\section{Einbinden von Quellen}\label{sec:quellen}
So bindet man Quellen in \LaTeX ein \cite{AllQuestionsSAP}.
Man kann auch eine Quelle mit Seitenzahl einbinden \cite[S. 420]{bianchiniPageRank2005}.
Hier wurde der Zitierstiel IEEE genutzt und mit BibTeX gearbeitet.
\clearpage

\section{Abbildungen einbinden}\label{sec:abbildungen}
In der Abbildung \ref{fig:nico} zeige ich das Internet in seiner vollendeten Form.
\begin{wrapfigure}{l}{7cm}
    \includegraphics[width=0.5\textwidth,angle=0]{abb/26z3fo.jpg}
    \caption[Bildbeschriftung für Abbildungsverzeichnis]{Bildbeschriftung unter der Abbildung (Quelle: \url{https://imgflip.com/})}
    \label{fig:nico}
\end{wrapfigure}
\blindtext[1]

Die Abbildung \ref{fig:latex} zeigt \dots
\begin{figure}[hb]
    \begin{subfigure}{6cm}
      \centering
      \includegraphics[width=6.5cm]{abb/1zfte27tn3q21.jpg}
      \caption{Caption text 1 (Quelle: \url{https://me.me/})}
    \end{subfigure}
    \begin{subfigure}{6cm}
      \centering
      \includegraphics[width=6.5cm]{abb/bucket.png}
      \caption{Caption text 2 (Quelle: \url{https://www.mememaker.net/})}
    \end{subfigure}
    \caption[Memes sind toll]{Memes sind toll}
    \label{fig:latex}
\end{figure}
\clearpage

% Literaturverzeichnis =======================================================
\addcontentsline{toc}{section}{Literaturverzeichnis} % Literaturverzeichnis im Inhaltsverzeichnis anzeigen
\renewcommand\refname{Literaturverzeichnis}
\printbibliography
\clearpage

\end{document}

