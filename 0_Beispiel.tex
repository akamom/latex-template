\section{Erste Ebene}\label{sec:ebene}
\blindtext
\subsection{Zweite Ebene}\label{sec:subebene}
\blindtext[1]
\subsubsection{Dritte Ebene}\label{sec:subsubebene}
Ich referenziere hier auf Kapitel \ref{sec:ebene}.
\blindtext[2]

\clearpage

\section{Umgebungen}\label{sec:umgebungen}
Hier sind einige Beispiele für Umgebungen aufgelistet.

\subsection{Umgebung für Listen}
\begin{itemize}
    \item Eintrag A
    \item Eintrag B
    \item \dots
\end{itemize}

\subsection{Umgebung für Tabellen}
Bei der Erstellung der Tabellen gibt es viele hilfreiche Tools im Internet
(\href{https://www.tablesgenerator.com/}{Table Generator}).

Dies Tabelle \ref{table:simple} wurde mit dem Table Generator erstellt.
\begin{table}[h!]
    \centering
    \caption{So muss es in der Studienarbeit aussehen}
    \label{table:simple}
    \begin{tabular}[t]{|l|l|l|}
        \rowcolor[HTML]{CBCEFB}
        \hline 
        A & B & C \\ \hline
        1 & 2 & 3 \\
        1 & 2 & 3 \\ \hline
    \end{tabular}
\end{table}

Die Tabelle \ref{table:multirow} besitzt sogar verbundene Zellen.
\begin{table}[h!]
    \centering
    \caption{Mit verbundenen Zellen}
    \label{table:multirow}
    \begin{tabular}[t]{|l|l|l|}
    \hline
                             & \multicolumn{2}{l|}{Eigenschaften} \\ \hline
    \multirow{3}{*}{Objekte} & Tag              & Name            \\ \cline{2-3} 
                             & Frage            & Datum           \\ \cline{2-3} 
                             & Author           & Name            \\ \hline
    \end{tabular}
\end{table}

\clearpage

\section{Einbinden von Quellen}\label{sec:quellen}
So bindet man Quellen in \LaTeX ein \cite{AllQuestionsSAP}.
Man kann auch eine Quelle mit Seitenzahl einbinden \cite[S. 420]{bianchiniPageRank2005}.
Hier wurde der Zitierstiel IEEE genutzt und mit BibTeX gearbeitet.

\clearpage

\section{Abbildungen einbinden}\label{sec:abbildungen}
In der Abbildung \ref{fig:nico} zeige ich das Internet in seiner vollendeten Form.
\begin{wrapfigure}{l}{8cm}
    \includegraphics[width=0.5\textwidth,angle=0]{abb/26z3fo.jpg}
    \caption[Bildbeschriftung für Abbildungsverzeichnis]{Bildbeschriftung unter der Abbildung\\ (Quelle: \url{https://imgflip.com/})}
    \label{fig:nico}
\end{wrapfigure}
\blindtext[1]

Die Abbildung \ref{fig:latex} zeigt \dots
\begin{figure}[hb]
    \begin{subfigure}{6cm}
      \centering
      \includegraphics[width=6.5cm]{abb/1zfte27tn3q21.jpg}
      \caption{Caption text 1\\ (Quelle: \url{https://me.me/})}
    \end{subfigure}
    \begin{subfigure}{6cm}
      \centering
      \includegraphics[width=6.5cm]{abb/bucket.png}
      \caption{Caption text 2\\ (Quelle: \url{https://www.mememaker.net/})}
    \end{subfigure}
    \caption[Memes sind toll]{Memes sind toll}
    \label{fig:latex}
\end{figure}